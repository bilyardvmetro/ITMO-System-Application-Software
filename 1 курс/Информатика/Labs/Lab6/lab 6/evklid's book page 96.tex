\documentclass[10pt, onecolumn]{article}

\usepackage{tikz}
\usepackage{tkz-euclide}
\usetikzlibrary{angles,quotes}
\usepackage[english,russian]{babel}
\usepackage[utf8]{inputenc}
\usepackage[T2A]{fontenc}
\usepackage[left=5mm, top=20mm, right=5mm, bottom=10mm, nofoot]{geometry}
\usepackage{tempora}
\usepackage{newtxmath}
\usepackage{multicol}
\usepackage{graphicx}

\date{}

\setlength{\columnsep}{-3cm}

\begin{document}

    % Команды с отрисовкой, чтобы не дублировать код
    \newcommand{\BD}{
        \begin{tikzpicture}
        
            \draw[line width=1mm, red] (0, 0) -- (1, 0);
            \filldraw[red] (0, 0) circle (1pt) node[scale=0.5, anchor=south] {B};
            \filldraw[red] (1, 0) circle (1pt) node[scale=0.5, anchor=south] {D};
            
        \end{tikzpicture}
    }
    
    \newcommand{\AC}{
        \begin{tikzpicture}
        
            \draw[line width=1mm, black] (0, 0) -- (1, 0);
            \filldraw[black] (0, 0) circle (1pt) node[scale=0.5, anchor=south] {A};
            \filldraw[black] (1, 0) circle (1pt) node[scale=0.5, anchor=south] {C};
            
        \end{tikzpicture}
    }
    
    \newcommand{\EF}{
        \begin{tikzpicture}
        
            \draw[line width=1mm, dotted] (0, 0) -- (1, 0);
            \filldraw[black] (0, 0) circle (1pt) node[scale=0.5, anchor=south] {E};
            \filldraw[black] (1, 0) circle (1pt) node[scale=0.5, anchor=south]{F};
            
        \end{tikzpicture}   
    }

    \newcommand{\CEF}{
        \begin{tikzpicture}[scale=0.3]
        
            \coordinate (C) at (2.83, -2.8);
            \coordinate (D) at (3.45, -1.99);
            \coordinate (E) at (-0.5,-1.6);
            \coordinate (F) at (0, 0);
            
            \draw pic[fill=magenta, angle radius=1cm] {angle = D--E--F};
            \draw pic[fill=green, angle radius=1cm] {angle = C--E--D};
            
            \draw node[scale=0.7, anchor=north] at (2.83, -2.8) {C};
            \draw node[scale=0.7, anchor=west] at (3.05, -1.65) {D};
            \draw node[scale=0.7, anchor=north] at (-0.5,-1.6) {E};
            \draw node[scale=0.7, anchor=south] at (0.5,1.6) {F};
            
        \end{tikzpicture}
    }

    \newcommand{\DEF}{
        \begin{tikzpicture}[scale=0.3]
            
            \coordinate (D) at (3.45, -1.99);
            \coordinate (E) at (-0.5,-1.6);
            \coordinate (F) at (0, 0);

            \draw pic[fill=magenta, angle radius=1cm] {angle = D--E--F};

            \draw node[scale=0.7, anchor=west] at (3.05, -1.65) {D};
            \draw node[scale=0.7, anchor=north] at (-0.5,-1.6) {E};
            \draw node[scale=0.7, anchor=south] at (0.5,1.6) {F};
            
        \end{tikzpicture}
    }

    \newcommand{\rightAngle}{
        \begin{tikzpicture}[scale=1]
        
            \coordinate (A) at (0, 1);
            \coordinate (B) at (0, 0);
            \coordinate (C) at (-1, 0);

            %\draw pic[angle radius=1cm] {angle = A--B--C};
            \draw[line width=0.3mm] (A) -- (B);
            \draw[line width=0.3mm] (B) -- (C);
            \tkzMarkAngle[line width=1mm](A,B,C);
            
        \end{tikzpicture}
    }

    
    \Large{
        \hspace{6cm} \emph{96} \hspace{2cm} КНИГА III ПРЕДЛ. IV. ТЕОРЕМА
    }
    \\
    \begin{multicols}{2}
        
        \begin{tikzpicture}[scale=0.8]
        
            % Окружность
            \draw[line width=1mm, magenta] (0, 0) circle [radius=4];
            
            % Линии
            \draw[line width=1mm, dotted] (0, 0) -- (253:1.73);
            \draw[line width=1mm, red] (-3.8, -1.3) -- (3.45, -1.99);
            \draw[line width=1mm] (2.83, -2.8) -- (-3.96, -0.4);

            % Задаём точки для того, чтобы по ним отрисовать углы
            \coordinate (A) at (-3.96, -0.4);
            \coordinate (B) at (-3.8, -1.3);
            \coordinate (C) at (2.83, -2.8);
            \coordinate (D) at (3.45, -1.99);
            \coordinate (E) at (-0.5,-1.6);
            \coordinate (F) at (0, 0);
            
            % Углы
            \draw pic[fill=magenta, angle radius=1cm] {angle = D--E--F};
            \draw pic[fill=green, angle radius=1cm] {angle = C--E--D};

            % Подписываем точки
            \filldraw[black] (-3.96, -0.4) circle (3pt) node[anchor=east]{A};
            \filldraw[black] (-3.8, -1.3) circle (3pt) node[anchor=east]{B};
            \filldraw[black] (2.83, -2.8) circle (3pt) node[anchor=west]{C};
            \filldraw[black] (3.45, -1.99) circle (3pt) node[anchor=west]{D};
            \filldraw[black] (-0.5,-1.6) circle (3pt) node[anchor=north]{E};
            \filldraw[black] (0, 0) circle (3pt) node[anchor=south]{F};
            
        \end{tikzpicture}
        
        \vfill\null
        \columnbreak

        \setlength{\columnsep}{-5cm}
        
        \begin{multicols}{2}
            \includegraphics[width=2.5cm, height=2.5cm ]{images/letter.png}
			
            \vfill\null
            \columnbreak
			
            \noindent сли \textit{в круге две прямые, не проходящие через центр, пересекаются, они не делят друг друга пополам.} 
        \end{multicols}
        \\
        Если одна из прямых проходит через центр, очевидно, она не может рассекать пополам другую прямую, не проходящую через центр.

        \\
        \vspace{0.5cm}
        
        \par Но если одна из прямых \AC или \BD не проходит через центр, проведём \EF из центра к точке их пересечения.

        \begin{center}
        
            Если \AC делится пополам, \EF $\perp$ ей (пр. III. \emph{3})

            $\therefore \CEF = \rightAngle$
            
            и если \BD делится пополам, \EF $\perp$ \BD (пр. III. \emph{3})

            $\therefore \DEF = \rightAngle;$
            
            \\
            \vspace{1cm}
            
            и $\therefore \DEF = \CEF ;$
            \\
            часть равна целому, что невозможно.
            
            \\
            \vspace{0.5cm}
            
            $\therefore$ \AC и \BD не делят друг друга пополам.
            
        \end{center}
        
        \begin{flushright}ч.т.д.\end{flushright}
        
    \end{multicols}
    
\end{document}
