\documentclass[12pt, onecolumn]{article}

\usepackage{tikz}
\usepackage[english,russian]{babel}
\usepackage[utf8]{inputenc}
\usepackage[T2A]{fontenc}
%\usepackage{float}
\usepackage{floatrow}
\usepackage[left=5mm, top=20mm, right=10mm, bottom=20mm, nofoot]{geometry}
\usepackage{tempora}
\usepackage{newtxmath}
\usepackage{multicol}
\usepackage{graphicx}

\date{}

\begin{document}
    \Large{
        \hspace{6cm} \emph{96} \hspace{2cm} КНИГА III ПРЕДЛ. IV. ТЕОРЕМА
    }\\
    \begin{multicols}{2}
        
        \begin{tikzpicture}
            % Окружность
            \draw[line width=1mm, magenta] (0, 0) circle [radius=4];
            % Линии
            \draw[line width=1mm, dotted] (0, 0) -- (253:1.73);
            \draw[line width=1mm, red] (-3.8, -1.3) -- (3.45, -1.99);
            \draw[line width=1mm] (2.83, -2.8) -- (-3.96, -0.4);

            % Углы
            %\filldraw[draw=lime, fill=lime] (1.05, -1.81) arc[radius=0.7, start angle=-289, end angle=-340] -- (-0.48,-1.61) -- cycle;
            %\filldraw[draw=magenta, fill=magenta] (0.1, 0.00005) arc[radius=1, start angle=-230, end angle=-350] -- (-0.48,-1.61) -- cycle;

            % Точки
            \filldraw[black] (-3.96, -0.4) circle (3pt) node[anchor=east]{A};
            \filldraw[black] (-3.8, -1.3) circle (3pt) node[anchor=east]{B};
            \filldraw[black] (2.83, -2.8) circle (3pt) node[anchor=west]{C};
            \filldraw[black] (3.45, -1.99) circle (3pt) node[anchor=west]{D};
            \filldraw[black] (-0.5,-1.6) circle (3pt) node[anchor=north]{E};
            \filldraw[black] (0, 0) circle (3pt) node[anchor=south]{F};
        \end{tikzpicture}
        
        \vfill\null
        \columnbreak
        
        \includegraphics[width=2cm, height=2cm ]{images/letter.png} сли \textit{в круге две прямые, не проходящие через центр, пересекаются, они не делят друг друга пополам.} \\[1cm]
        
        Если одна из прямых проходит через центр, очевидно, она не может рассекать пополам другую прямую, не проходящую через центр.
        
        \par Но если одна из прямых ac или bd не проходит через центр, проведём ef из центра к точке их пересечения.
        
    \end{multicols}
\end{document}
